\documentclass{article}
\usepackage{graphicx} % Required for inserting images
\newtheorem{fact}{Fact}
\usepackage{amsmath}
\usepackage{amsthm}
\usepackage{array}
\usepackage[margin=1in]{geometry}

\title{CS561 HW10}
\author{Ryan Scherbarth}
\date{November 2024}

\begin{document}

\maketitle

\begin{enumerate}

%%
%%% Begin Q1
%%
\item \textbf{ Prove the following fact, from Slide 114, amortized lecture, about the Union-Find data structure using PC-UNION:
\begin{itemize}
\item For any set $X$, size$(X) \geq 2^{\text{rank(leader($X$)}}$
\end{itemize}
Prove this by induction on the size of $X$. Don't forget to include the BC, IH, and IS. 
} \\
\newline
% \textbf{Base Case:} \\
% If a set $X$ has a single element, $rank(leader(X)) = 0$, and $size(X) = 1 = 2^{0}$. Therefore, our base case holds. \\

% \textbf{Inductive Hypothesis:} \\
% For any $J$ s.t. $rank(leader(J)) \leq k$, $size(J) \geq 2^{rank(leader(J))}$. \\

% \textbf{Inductive Step:}
% Let $S_1$ and $S_2$ be two sets, s.t. both meet the requirements of the inductive hypothesis. There are two cases that can take place as part of our inductive step. \\
% \newline
% \textbf{Case 1:} $rank(S_1) == rank(S_2)$ \\
% In the case where the rank of both of the sets is equal; 
% \newline
% $x_1 = rank(leader(S_2))$ \\
% $x_2 = rank(leader(S_1))$ \\
% \newline
% s.t. $x_1 \leq x_2$. Once we do a $UNION$, the rank will only increase when $x_1 = x_2$; and in this case we increase the rank by 1. \\
% \newline
% Since our hypothesis tells us $size(S_1) \geq 2^{x_1}$ and $size(S_2) \geq 2^{x_2}$, the size of the new $X = S_1 \cup S_2$ will give $size(X) \geq 2^{x_1} + 2^{x_2} = 2^{x_1 + 1}$. Therefore, by inductive hypothesis, it is true that any set $X$, $size(X) = 2^{rank(leader(X))}$. \\

% \textbf{Case 2:} $rank(S_1) \neq rank(S_2)$ \\
% In the case where the rank of both of the sets is not equal, we will define our sets s.t. $S_1 > S_2$. When we perform a $UNION$, the $UNION-FIND$ ensures that the leader of the resulting set $X = S_1 \cup S_2$ will be the leader of the set with a higher rank. \\
% \newline 
% Since $S_1$ and $S_2$ are defined as having satisfied the inductive hypothesis, we can say $size(S_1) \geq 2^{rank(S_1)}$ and $size(S_2) \geq 2^{rank(S_2)}$. \\
% \newline 
% In this case, $size(X) = size(S_1) + size(S_2) \geq 2^{rank(S_1)} + 2^{rank(S_2)}$. Since we know $rank(S_1) > rank(S_2)$, adding $S_2$ to $S_1$ doesn't change the rank of $S_1$. \\
% \newline 
% Therefore, we can conclude that $size(X) \geq 2^{rank(S_1)}$
\textbf{Base Case:} \\
% If a set $X$ has a single element, $rank(leader(X)) = 0$, and $size(X) = 1 = 2^{0}$. Therefore, our base case holds. \\
If $x=0$, then $S$ is a single element. For any set $S$ with $rank(leader(S)) = 0$, then $size(S) = 2^0 = 1$, so the base case holds true. \\

\textbf{Inductive Hypothesis:} \\
% For any $J$ s.t. $rank(leader(J)) \leq k$, $size(J) \geq 2^{rank(leader(J))}$. \\
Assume that for any $J$ s..t $size(J) < x$, it holds that $size(J) \geq 2^{rank(leader(J))}$. \\

\textbf{Inductive Step:} \\
There are two cases that can occur when performing a UNION on two sets, $S_1$ and $S_2$, each satisfying the inductive hypothesis; 

\textbf{Case 1: } $size(S_1) = size(S_2)$ \\
If $rank(leader(S_1)) = rank(leader(S_2))$, the union of $S_1$ and $S_2$ will increase the rank by exactly 1. \\
\newline 
We will now have $rank(leader(S)) = rank(leader(S_1)) + 1$, \\
therefore $size(S) \geq 2^{rank(leader(S_1)) + 2^{rank(leader(S_2))}} \geq 2^{rank(leader(S))}$. \\
\newline
We can then confirm the case holds that $size(S) \geq 2^{rank(leader(S))}$. \\

\textbf{Case 2: } $size(S_1) \neq size(S_2)$ \\
Let $x_1 = rank(leader(S_1))$ and $x_2 = rank(leader(S_2))$ and lets define them s.t. $S_1 \geq S_2$. \\
\newline 
When performing a union on two equal height sets, the rank will remain the same. Since $x_1 > x_2$, we can confirm that the result will be equal to the size of $x_1$. \\
\newline 
Therefore $size(S) = size(S_1) \geq 2^{x_1}$ and $x_1 = rank(leader(S_1))$.







%%
%%% Begin Q2
%%
\newpage
\item \textbf{ Professor Moe conjectures that for any connected graph $G$, the set of edges } \\
$\{ (u,v) : \text{there exists a cut } (S, V-S) \text{ such that } (u,v) \text{ is a light edge crossing} (S, V-S) \}$ \\
\textbf{always forms a minimum spanning tree. Give a simple example of a connected graph that proves him wrong.} \\

Say we have the graph $V$ s.t. we have 4 vertexes; $A, B, C, D$, and 4 edges; $\{ (A,B), (B,C), (C,D), (D,A), (B,D) \}$ with corresponding weights as $1, 2, 3$, and $4$. Our graph is in the form of a straight line containing the 4 vertexes, each of which are connected and then we have a final connection looping around between $A$ and $D$, and an edge between $B$ and $D$ with weight $1$. \\
\newline 
In this case, $\{ (u,v) : \text{there exists a cut } (S, V-S) \text{ such that } (u,v) \text{ is a light edge crossing} (S, V-S) \}$ results in the set $\{ (A,B), (B,C), (C,D), (B,D) \}$. We define the minimum spanning tree $M$ as a subset of some larger graph, $T$, s.t. $M$ includes all vertices of $T$, and forms a connected, acyclic sub-graph with the minimal value to the weight function, $w(T) = \sum_{e \in E_T} w(e)$. \\
\newline 
Given our definition we can trivially prove that the sub-graph generated from Professor Moe's conjecture does not satisfy the criteria for a minimum spanning tree. With this example, our graph returned does meet the first criteria of a minimum spanning tree by minimizing the weight function, $w$, however it contains a cycle between $A,C,D$. This fails the first condition of a MST in that it is not an acyclic sub-graph of $T$, and therefore disproves Professor Moe's conjecture by counter example.  \\
\newline







%%
%%% Begin Q3
%%
\item \textbf{Consider a connected graph $G = (V,E)$. Call a subset of edges, $F$, a $cycle$ $cover$ if every cycle in $G$ contains at least one edge in $F$. In other words, removing the edges of $F$ from $G$ results in an acyclic graph. You want to find a cycle cover, $F$ of $G$, with minimum weight, i.e., the sum of the weight of all edges in $F$ is minimized over all cycle covers.} \\
\textbf{Give an efficient algorithm to solve this, and give the runtime of your algorithm as a function of $n = |V|$, and $m = |E|$. Hint: Think about the maximum-weight spanning tree problem.} \\
\newline 

Our algorithm will start my finding the $MST$ of $G$ using Kruskal's algorithm. After we've created the $MST$, the set of nodes that is not included in this $MST$ will give us our cycle cover for $F$. We can get this by taking the set difference $G - MST$. \\
\newline 
When we form the $MST$ we are selecting the maximum weight edges at each iteration. A side-effect of this is that the remaining edges satisfy the requirements as defined in the problem as a cycle cover, having the minimum total weight. Removing any edge from the $MST$ also creates a cycle, so we can conclude that the remaining edges cover all cycles in the graph. \\
\newline
The runtime of this algorithm is determined by how long it takes to find the $MST$ of the set of nodes. To run Kruskal's algorithm we will spend $O(E \log(E))$ time to sort the edges, which will ultimately bound our runtime to $O(m \log(m))$. 








%%
%%% Begin Q4
%%
\newpage
\item \textbf{ Professor Matsumoto conjectures the following converse of the safe edge theorem: } \\
\textbf{ Let $G = (V,E)$ be a connected, undirected, weighted graph, with weight function $w$. Let $A$ be a subset of $E$ that is included in some minimum spanning tree of $G$. Let $(S, V-S)$ be any cut of $G$ that respects $A$, and let $(u,v)$ be a safe edge for $A$ that crosses $(S, V-S)$. Then $(u,v)$ is a light edge for the cut. Is this conjecture true? If so, prove it. If not, give a counterexample. } \\

Let us define a graph, $G$, consisting of vertexes; $V = \{ a,b,c \}$, and edges $E = \{ (a,b), (b,c), (c,a) \}$ with corresponding weights 1, 2, and 3. \\
\newline 
Creating the MST for this graph will use the edges $(a,b)$ and $(b,c)$ with a total weight of 3. If we let $A = \{ (a,b) \}$ which is a part of our MST, we can then make a cut $(S, V-S) = (\{a\}, \{b,c\})$, which respects $A$ since $(a,b) \in A$ crosses this cut. \\
\newline 
In this case, our safe edge becomes $(b,c)$. This is the only possible safe edge that crosses $(S, V-S)$. We see this to be a counter-example as the edge $(b,c)$ is not the lightest edge crossing the cut, the lightest is actually edge $(a,b)$. Therefore, this conjecture is not valid. \\
\newline







%%
%%% Begin Q5
%%
\item \textbf{ Prove that if an edge $(u,v)$ is in some minimum spanning tree for a graph, $G$, then $(u,v)$ is a light edge crossing some cut in $G$. } \\

Let $T$ be a MST containing the edge $(u,v)$. We define a cut $(S, V-S)$ s.t. $u \in S$ and $v \in V-S$ separating $u$ and $v$. \\
\newline 
The safe edge theorem tells us that an edge $e$ is safe for set $A$, where $A$ is a subset of edges forming an MST, if it is a light edge crossing a cut that respects $A$. Since $T$ is an MST, every edge in $T$ must be a safe edge for some set of edges $A$ with respected to $T$. \\
\newline 
Since the cut $(S, V-S)$ is defined to have separated $u$ and $v$, then edge $(u,v)$ must be the lightest edge across this cut that does not violate any of the rules of a MST. \\
\newline 
Therefore, using the safe edge theorem we can prove that any edge in a MST satisfies the property of being a light edge for some cut in $G$.


\end{enumerate}

\end{document}